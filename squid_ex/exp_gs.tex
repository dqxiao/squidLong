\subsection{Results on uncertain graph statistic preserving}
In the set of experiment, we focus on evaluating their performance concerning statistic preserving. 
The evaluation includes two groups of graph statistics. 
The first group includes node separation statistics that quantify the interconnectivity and density of the overall graph.   
This group includes metrics such as Graph Diameter, Average  Path  Length~\footnote{In particular, we use Hyper ANF~\cite{Boldi_Rosa_Vigna_2011} to approximate shortest path-based statistics.}, and Reliability. 
The second group includes degree-based statistics such as Average  Node Degree and Degree Distribution.
These topological statistics that characterize how degrees are distributed among nodes. 
Our results are highly consistent across our pool of graph statistics.
For brevity, we report only Reliability, Node Degree as their representatives. 

\textbf{Node Separation Statistic.} 
In Figure~\ref{fig:rd}, we plot how the connectivity distortion increases as the anonymity level $k$ increases. 
As expected, larger $k$ introduced more significant connectivity distortion, because more noise was added to achieve the desired level of anonymity.   
These observations are consistent across biological and social uncertain graphs.
We report the reliability discrepancy compared with ones of the original uncertain graph. 

We can see that {\methodName} achieve much better result than CA and RA, due to the integration of possible world semantics and uncertainty-aware search strategy. 
For example, in all the dataset ($k=300$), the reliability discrepancy introduced by {\methodName} is well below 0.1 whereas the ones added by CA is below 0.2. The reliability discrepancy introduced by RA on PPI, BK, DBLP is around 0.2, 0.2, 0.4 respectively. We also observe that as the size of graph increases (PPI $\rightarrow$ DBLP), the performance gap becomes larger and larger.

Note that CA and RA schemes deteriorate data utility due to the disregarding the possible world semantics. 
For example, in the case, $k=100$ (weak privacy guarantee required little noise), CA and RA incur relatively large connectivity distortion. 
The representative extraction step of RA introduces noise and results in cumulative errors in the anonymization step. Consequently, sanitized results differ from the original ones. 
The CA scheme fails to reflect the connectivity of uncertain graph correctly. Therefore, it produces inferior results even with the focused anonymization strategy. 


\textbf{Degree-based Statistic.}~Figure~\ref{fig:dd} shows the error of Node Degree values on PPI, BK and DBLP compared to their sanitized outputs.  
In comparison, the quality as generated of our {\methodName} is much better.  
In the extreme case(DBLP, $k=300$), the degree error is $0.2$ instead of $1.08$ imposed by the two-phase anonymization scheme in the same level of obfuscation.  
As with previous experiments, the performance gap increases as the graph size increases.

\textbf{Summary.} 
Our experimental results show that sanitized outputs generated by {\methodName} exhibit structural features close to those of their original uncertain graphs. 
They show that we are able to balance the utility and privacy for sharing uncertain graphs effectively. 
The result is encouraging because we can eliminate the noise by moving from the reliability model to a more accurate graph model incorporating with the possible world semantics.