\section{Related Work}
\label{sec:relatedWork}
A significant amount of prior work has been done on protecting the privacy of network datasets.
The comprehensive survey is out of the scope of this paper. 
Here, we briefly summarize related work and clarify our privacy goal. 

\textbf{Syntactic Privacy.}~~
Early works on privacy-preserving network publishing focus on developing anonymization techniques. 
Many of them modify the graph structure in subtle ways that guarantee privacy but keep much of graph structure for release. 
The released graph is available for all the analysis tasks. 
These approaches usually provide privacy protection against specific de-anonymization attacks. 
Most of them leverage syntactic privacy models derived from $k$-anonymity~\cite{Sweeney:2002:KAM:774544.774552} which requires creating $k$ same entities ({\eg} neighborhoods, degree nodes) to blend victims. 

Related anonymization methods can be classified into four main categories: (1) Clustering-based generalization~\cite{Hay_Anonymizing_2007,Bhagat_Class_2009,hay2010resisting}; (2)~{\em Edge modification}~\cite{Liu_Towards_2008, Zhou_Preserving_2008, Zou:2009, Wang2011, Wu_k_2010, Skarkala_Privacy_2012}; 
(3)~{\em Edge randomization}~\cite{Liu_Privacy_2009,Ying_Randomizing_2008, Ninggal_Utility_2015};
and~(4)~{\em Uncertainty semantic-based modifications} which add uncertainty to some edges and thus converting the deterministic graph to an uncertain version for anonymity~\cite{Boldi_Injecting_2012, Nguyen_Anonymizing_2015}. 

In the first category, Hay {\etal}~\cite{Hay_Anonymizing_2007} proposed to generalize a network by clustering nodes and only publish the hyper-graph ($\#$ of nodes in each partition with $\#$ of edges within and across partitions). Campan {\etal}~\cite{Campan2008} studied the attributed graph case with a similar solution. 
Cohen {\etal}~\cite{Cohen2013} presented a sequential clustering algorithm with better utility preserving. While, Cormode{\etal} \cite{Bhagat_Class_2009} payed attention to attributed-based matching attack. To this end, their method marks the mapping by clustering the nodes and corresponding real world entities into groups. 

In the second category, Liu {\etal}~\cite{Liu_Towards_2008} focused on resisting degree-based entity re-identification attacks. They propose to add and delete edges to pursue $k$-degree anonymity. Zhou {\etal}~\cite{Zhou_Preserving_2008} consider stronger re-identification attack based on radius-one subgraph. Zhou~{\etal}~\cite{Zou:2009} assume that the adversary knows the compute graph. Their algorithms use edge addition and deletion to make graph $k$-Automorphism.  

In the third category, Hay {\etal} \cite{Liu_Privacy_2009} study the use of random perturbation for identity obfuscation. They consider the basic degree-based re-identification of nodes. Besides, they propose to quantify the level of anonymity that is provided for the given node $v$ in the real network by the perturbed graph as the inverse of the maximum of the belief probability $Pr(v|u)$. Ying {\etal} \cite{Ying_Randomizing_2008} compare random perturbation methods to the method of $k-$degree anonymity. Their experiments over two datasets (Enron and Pollblogs) show the deterministic edge modification methods for $k$-degree anonymity preserve the graph structure better than random perturbation methods.

The uncertainty semantic-based approaches are known as the state-of-art ones because of their excellent privacy-utility trade-off, brought by the fine-grained perturbation leveraging the uncertain semantics. 
Our method belongs to the fourth category while for the probabilistic context. 

 
\textbf{Differential Privacy.}~~Another option is to apply {$\epsilon$}-differential privacy policy to providing privacy guarantee. It roughly falls into two directions. The first direction aims to release specific differentially private mining results, such as degree distributions, sub-graph counts, and frequent graph patterns~\cite{Xiao_Differentially_2014,Day:2016}. These methods only publish query result. However, there are many situations in which answering statistical queries simply does not achieve the purpose of sharing the graph.  
The second direction aims to share the meaningful graph~\cite{Sala_Sharing_2011}. Most research in this direction projects an input graph to dK-series and ensures differential privacy on dK-series statistics. Later, private statistics are then either fed into generators or MCMC process to generate a fit synthetic graphs. While current techniques are still inadequate to provide desirable data utility for many graph mining tasks. 

\emph{All the methods target at providing privacy guarantee to the deterministic graph. The uncertain scenario is unexplored.}

\subsection{Our Privacy Goal}
As ever discussed, existing schemes fail to provide utility guarantee in the uncertain scenario. 
In this work, we try to move this line of research one step forward from the deterministic context to a broader probabilistic context. 
We believe the anonymization process needed to be specially optimized to uncertain graphs.
Our work seeks a solution to share meaningful uncertain graphs while preserving privacy.

There is a widespread belief that differential privacy and its offsprings are immune to various privacy attacks. It offers a guarantee bound $\epsilon$ on the loss of privacy due to the data release~\cite{Sala_Sharing_2011,Xiao_Differentially_2014}. However, there is no clear way to set general policy for choosing the privacy parameter $\epsilon$ for sufficient privacy guarantee~\cite{lee2011}. Its implications and impacts on the risk of disclosure in practice heavy depend on data detail. Thus, differential privacy is difficult to apply in practice. 

In contrast, the notion of syntactic privacy can be defined and understood based on the data schema. And, its parameters have a clear privacy meaning that can be understood independent of the actual data. Moreover, they have a clear relationship to the privacy regulation of individual identifiability (e.g., General Data Protection Regulation GDPR). 
Hence, we focus on sharing uncertain graphs with syntactic anonymity. 